\documentclass[10pt,brazil,english]{article}
\usepackage{amsfonts}
\usepackage{infocomp}
\usepackage{times}
\usepackage{amsmath}
\usepackage{amssymb}
\usepackage[T1]{fontenc}
\usepackage[english, portuguese]{babel}
\addto\captionsportuguese{
\renewcommand{\figurename}{Figura}
\renewcommand{\tablename}{Tabela}
\renewcommand{\refname}{REFER\^{E}NCIAS}
}
\usepackage[utf8]{inputenc}
\usepackage{multirow}
\usepackage{lscape}
\usepackage{rotating}
\usepackage{setspace} % espacamento entre linhas
\usepackage[table,xcdraw]{xcolor}
\usepackage{scalefnt}
\usepackage{graphicx}
\usepackage{hyperref}
\usepackage{subfigure}
\usepackage{enumerate}
\usepackage{caption}
\usepackage[sort,compress]{cite}
\usepackage[alf,abnt-repeated-author-omit=yes,abnt-etal-list=0]{abntex2cite}	% Citações padrão ABNT
%%%%%%%%%%%%%%%%%%%%%%%%%%%%%%%%%%%%%%%%%%%%%%%%%%%%%%%%%
\usepackage{fancyhdr}
\usepackage{mathtools}
\setcounter{page}{1}
\fancyhead{ }
\lhead{}
\chead{\footnotesize A PREENCHER}
\rhead{}
\cfoot{}
\rfoot{\thepage}%Direita do Rodapé
\renewcommand{\headrulewidth}{1pt}% Traço horizontal no cabeçalho

%%%%%%%%%%%%%%%%%%%%%%%%%%%%%%%%%%%%%%%%%%%%%%%%%%%%%%%%%

\usepackage{rangecite}

%\hyphenation{po-pu-la-ri-za-ção re-gis-tros do-mi-na-do-ra vio-la pe-ram-bu-lam dou-tri-na-ria-men-te co-nhe-ce-rem Ad-mis-tra-ção fa-bri-car so-cie-da-de in-fe-rio-res vee-men-te-men-te si-tua-ção pon-tuais}

\sloppy
\renewcommand{\captionfont}{\footnotesize}
\renewcommand{\captionlabelfont}{\footnotesize \bfseries}
\newtheorem{exemplo}{Exemplo}
\title{A PREENCHER}

\address{
$^{1}$Fundação Getulio Vargas (FGV)}

\author{Lucas Emanuel Resck Domingues$^{1}$}

\selectlanguage{english}

\abstract{To be filled.}

\keywords{To. Be. Filled.}

\selectlanguage{brazil}

\resumo {A preencher.}

\palchaves{A. Preencher.}

\begin{document}
    \pagestyle{fancy} % CABECALHOO
    
    \maketitle
    \newpage

    \newtheorem{theorem}{Teorema}
    
    \section{\uppercase{Introdução}}
    
    \section{\uppercase{Modelo Bak-Sneppen}}

        \citeonline{bak1993punctuated} introduziram o modelo que chamarei de Bak-Sneppen.
        Seu funcionamento é descrito de forma direta por \citeonline{khouri2013estudos}:

        \renewcommand{\theenumi}{\roman{enumi}} 
        \begin{enumerate}
            \item $N$ espécies são distribuídas em um grafo circular;
            \item A cada espécie, é associado um número, chamado \textit{\textbf{fitness}}, que é uma amostra de uma distribuição uniforme em $[0, 1]$;
            \item Em cada "rodada", o indivíduo com o menor \textit{fitness} tem seu número sorteado novamente;
            \item O mesmo se repete para seus dois vizinhos.
        \end{enumerate}

        O modelo é relativamente simples, e no artigo é sugerido que a distribuição dos \textit{fitness} converge para $(f^*, 1]$, com $f^* \approx 2/3$.

    \section{\uppercase{Modelo GMS}}

        \citeonline{guiol2009stochastic} se basearam no modelo Bak-Sneppen para criar outro modelo, que \citeonline{khouri2013estudos} chama de GMS (as iniciais dos autores). Chamaremos assim neste trabalho. Ele é descrito:

        \renewcommand{\theenumi}{\roman{enumi}} 
        \begin{enumerate}
            \item O modelo é discreto e se inicia no conjunto vazio;
            \item A cada etapa do processo, uma espécie nasce (com probabilidade $p$) ou uma espécie morre (com probabilidade $q = 1 - p$) (se o conjunto de espécies é vazio, ele se mantém constante);
            \item Cada espécie tem associada a si um número \textit{fitness} amostrado de uma uniforme $[0, 1]$;
            \item Quando uma espécie morre, a escolhida é aquela com menor \textit{fitness} para morrer.
        \end{enumerate}

        Os autores apresentam alguns resultados. Seja $p$ tal que $p \in (1/2, 1)$ e tomemos $$f_c = \frac{1 - p}{p}\textrm{.}$$
        Então $f_c \in (0, 1)$.
        Consideremos $L_n$ e $R_n$ as espécies que estão vivas no tempo $n$ que são menores e maiores que $f_c$, respectivamente. Ou seja, $L_n \subseteq (0, f_c)$ e $R_n \subseteq (f_c, 1)$. A notação $|A|$ indica a cardinalidade de $A$. Então:

        \begin{theorem} \cite{guiol2009stochastic}
            \renewcommand{\theenumi}{\alph{enumi}} 
            \begin{enumerate}
                \item $|L_n|$ é uma cadeia de nascimento-morte recorrente nula. Em particular, $L_n$ é vazio infinitas vezes com probabilidade um.
                \item Se $f_c < a < b < 1$, então $$\lim_{n \to \infty} \dfrac{1}{n} |R_n \cap (a, b)| = p(b - a) \textrm{.}$$
            \end{enumerate}
        \end{theorem}

        São resultados muito fortes, porém \citeonline{guiol2009stochastic} resume dizendo que assintoticamente a distribuição das espécies se dá uniformemente em $(f_c, 1)$.

        \subsection{Modelo a tempo contínuo}
        
        \subsection{Modelo a tempo discreto}
    
    \section{\uppercase{Implementação}}

    \section{\uppercase{Resultados}}

    \section{\uppercase{Discussão}}

    \bibliography{Referencias}
    
\end{document} 